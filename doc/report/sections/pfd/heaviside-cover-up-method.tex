\subsection{Heaviside Cover-Up Method}

This method only applies to  distinct  roots of the denominator. However, there
is a workaround shown later, making it the method of choice.

\subsubsection{Third Order Example}

\begin{equation}
    T(s) = \frac{b_0 + b_1s + b_2s^2}{(s-p_1)(s-p_2)(s-p_3)}
         = \frac{A_1}{s-p_1} + \frac{A_2}{s-p_2} + \frac{A_3}{s-p_3}
\end{equation}

To determine $A_1$ by the  cover-up  method,  we  mentally  remove  the  factor
$(s-p_1)$ on the left hand side, and set $s=p_1$:

\begin{equation}
    T(s) = \left.\frac{b_0 + b_1s + b_2s^2}{\cancel{(s-p_1)}(s-p_2)(s-p_3)}\right|_{s=p_1}
         = \frac{b_0 + b_1p_1 + b_2p_1^2}{(p_1-p_2)(p_1-p_3)} = A_1
\end{equation}

Similarly, to determine $A_2$, we mentally remove the  factor $(s-p_2)$ and set
$s=p_2$:

\begin{equation}
    T(s) = \left.\frac{b_0 + b_1s + b_2s^2}{(s-p_1)\cancel{(s-p_2)}(s-p_3)}\right|_{s=p_2}
         = \frac{b_0 + b_1p_2 + b_2p_2^2}{(p_2-p_1)(p_2-p_3)} = A_2
\end{equation}

$A_3$ is calculated the same way:

\begin{equation}
    T(s) = \left.\frac{b_0 + b_1s + b_2s^2}{(s-p_1)(s-p_2)\cancel{(s-p_3)}}\right|_{s=p_3}
         = \frac{b_0 + b_1p_2 + b_2p_2^2}{(p_3-p_1)(p_3-p_2)} = A_3
\end{equation}

\subsection{Extension to Multiple Roots}

In the case of repeated roots, the method above will only yield the coefficient
for the most significant power. This can be  fixed  by  first factoring out the
repeated roots, performing the decomposition on  the inner fraction, fractoring
in the roots again, and performing a second decomposition:

\begin{align}
  T(s) &= \frac{b_0 + b_1s}{(s-p_1)^2(s-p_2)} \\
       &= \frac{1}{s-p_1}\left(\frac{b_0 + b_1s}{(s-p_1)(s-p_2)}\right) \\
       &= \frac{1}{s-p_1}\left(\frac{A_1}{s-p_1} + \frac{A_2}{s-p_2}\right) \\
       &= \frac{A_1}{(s-p_1)^2} + \frac{A_2}{(s-p_1)(s-p_2)} \\
       &= \frac{A_1}{(s-p_1)^2} + \frac{A_3}{s-p_1} + \frac{A_4}{s-p_2}
\end{align}

